\usepackage[utf8]{inputenc}	% arquivos LaTeX em Unicode (UTF8)
\usepackage{float}
\usepackage[T1]{fontenc}
\usepackage[brazil]{babel}
\usepackage{multicol}
\usepackage{url}
\usepackage{booktabs}

\usepackage{courier}			% Verbatim, Listings, etc

% inclusão de figuras em PDF, PNG, PS, EPS
\usepackage{graphicx}

% Inclusão de códigos
%\usepackage{minted}
%\setminted{
%    breaklines=true,
%    autogobble,
%    frame=single,
%}

\usepackage{listings}
\lstset{language=c}
\lstset{basicstyle=\ttfamily\footnotesize,commentstyle=\textit,stringstyle=\ttfamily}
\lstset{showspaces=false,showtabs=false,showstringspaces=false}
\lstset{numbers=left,stepnumber=1,numberstyle=\tiny}
\lstset{columns=flexible,mathescape=true}
\lstset{frame=single}
\lstset{breaklines=true} % adicionado Odair M 2022-09-04
\lstset{inputencoding=utf8,extendedchars=true}
\lstset{literate={á}{{\'a}}1   {ã}{{\~a}}1 {à}{{\`a}}1 {â}{{\^a}}1
                 {Á}{{\'A}}1   {Ã}{{\~A}}1 {À}{{\`A}}1 {Â}{{\^A}}1
                 {é}{{\'e}}1   {ê}{{\^e}}1 {É}{{\'E}}1 {Ê}{{\^E}}1
                 {í}{{\'\i}}1  {Í}{{\'I}}1
                 {ó}{{\'o}}1   {õ}{{\~o}}1 {ô}{{\^o}}1
                 {Ó}{{\'O}}1   {Õ}{{\~O}}1 {Ô}{{\^O}}1
                 {ú}{{\'u}}1   {Ú}{{\'U}}1
                 {ç}{{\c{c}}}1 {Ç}{{\c{C}}}1 }

% formatação de algoritmos
\usepackage{algorithm}
\usepackage{algpseudocode}

%\IfLanguageName{brazilian}{\floatname{algorithm}{Algoritmo}}{}
\renewcommand{\algorithmiccomment}[1]{~~~// #1}

\algnewcommand\algorithmicforeach{\textbf{for each}}
\algdef{S}[FOR]{ForEach}[1]{\algorithmicforeach\ #1\ \algorithmicdo}

% ------------------------------------------------------------------------------

% formatação de bibliografia
\usepackage{natbib}		% bibliografia no estilo NatBib
\renewcommand{\cite}{\citep}	% \cite deve funcionar como \citep

% ------------------------------------------------------------------------------

% fontes adicionais
\usepackage{amsmath}		% pacotes matemáticos
\usepackage{amsfonts}		% fontes matemáticas 
\usepackage{amssymb}		% símbolos 

% ------------------------------------------------------------------------------
% pacotes diversos
%\usepackage{alltt,moreverb}	% mais comandos no modo verbatim % comentado Odair 2022-09-17 erro do verbatim
\usepackage{lipsum}		% gera texto aleatório (para os exemplos)
\usepackage{currfile}		% infos sobre o arquivo/diretório atual
%\usepackage[final]{pdfpages}	% inclusão de páginas em PDF
\usepackage{longtable}		% tabelas multi-páginas (tab símbolos/acrônimos)

% ------------------------- Commands -----------------------------------------
\newcommand\FourQuad[4]{%
    \begin{minipage}[b][.35\textheight][t]{.47\textwidth}#1\end{minipage}\hfill%
    \begin{minipage}[b][.35\textheight][t]{.53\textwidth}#2\end{minipage}\\[-1.5em]
    \begin{minipage}[b][.35\textheight][t]{.47\textwidth}#3\end{minipage}\hfill
    \begin{minipage}[b][.35\textheight][t]{.53\textwidth}#4\end{minipage}%
}


% Draw diagram in tex
\usepackage{tikz}
% Note the default value for the first
% argument is provided by [This is a box]
\newcounter{phases}[section]
\newenvironment{phases}[2][]
    {\begin{center}\refstepcounter{phases}
    Example \thephases: #1\\[1ex]
    \begin{tabular}{|p{0.9\textwidth}|}
    \hline\\
    #2\\[0.5ex]
    }
    { 
    \\\\\hline
    \end{tabular} 
    \end{center}
    }
\usepackage[dvipsnames]{xcolor}
\newcommand{\red}[1]{\textcolor{red}{#1}}
\newcommand{\blue}[1]{\textcolor{blue}{#1}}
\newcommand{\green}[1]{\textcolor{Green}{#1}}
